\documentclass[11pt]{article}
\usepackage{geometry,url, hyperref}
\geometry{margin=1in, centering}
\usepackage[utf8]{inputenc}
\usepackage{multicol, multirow}
\usepackage{fancyhdr, lastpage, bbding, pmboxdraw, color, soul, booktabs, graphicx}
\usepackage{caption}
\usepackage{lmodern}
\usepackage[T1]{fontenc}  % Ensure proper encoding
\usepackage{titlesec}

\titleformat{\section}
  {\normalfont\large\bfseries\sffamily}{\thesection}{1em}{}

\titleformat{\subsection}
  {\normalfont\normalsize\bfseries\sffamily}{\thesection}{1em}{}
  
\hypersetup{
    colorlinks=true,
    linkcolor=blue,
    filecolor=magenta,      
    urlcolor=blue,
    citecolor=blue,
}

\pagenumbering{arabic}

\begin{document}

\fontfamily{lmss}\selectfont

\begin{center}
\textbf{\huge EC 380: International Economic Issues}\\ 
\vspace{0.1in}
University of Oregon\\ 
Department of Economics\\ 
\end{center}
\vspace{-.2in}
\begin{center}\begin{tabular}{llll}
\toprule
    \textbf{Instructor:} & \footnotesize{Jose Rojas-Fallas} & \textbf{Class Location:}  & {\footnotesize Straub 245} \\ 
    \textbf{Email:} & \footnotesize{\href{mailto:jrojas2@uoregon.edu}{jrojas2@uoregon.edu}} & \textbf{Class Day/Time:} & \footnotesize{Mon \& Wed: 2:00 - 3:20 pm} \\
    \textbf{Office Hours:} & \footnotesize{Wed 09:00 - 11:00 am} & \textbf{Office:} & \footnotesize{PLC 525}\\
\bottomrule
\end{tabular}
\end{center}

\bigskip

\section*{COURSE SUMMARY}

\subsection*{DESCRIPTION}
The objective of this course is to examine international economic issues using the tools developed in the principles of economics courses. 
Following Paul Krugman's advice: 
``The problem is that most of what a student is likely to read or hear about international economics is nonsense. 
(...) The most important thing to teach our undergrads about trade is how to detect that nonsense."
The course is presented in two sections: INTERNATIONAL TRADE \& INTERNATIONAL FINANCE.
Overall, we will learn the theoretical explanations for trade patterns that include technological, resource, and competitiveness differences, 
how although countries benefit from trade, some groups within a country should be expected to lose from trade, 
how examine policies such as tariffs, quotas, and free-trade agreements, 
and how trade in financial instruments is structured and spillover effects from these. 

\subsection*{PREREQUISITES}
The prerequisite for this course is EC 201: Principles of Microeconomics. 

\section*{LEARNING OBJECTIVES}
\begin{itemize}
    \item Develop fundamental economic models that explain why countries trade with one another
    \item Analyze the effects of international trade on specific groups within a country as well as the country as a whole
    \item Discuss how government policies influence international trade and how these policies influence special interests 
    \item Learn how international financial markets are structured and how currencies across countries relate
    \item Develop an understanding of how a country's individual macroeconomic situation (e.g., inflation and unemployment) influences other countries through foreign exchange markets
    \item Investigate how government policies might help address objectives of full employment and low inflation in a world with trade
\end{itemize}

\section*{LEARNING MATERIAL:}
\begin{itemize}
    % TALK ABOUT TEXTBOOK HERE
    \item \textbf{Textbook:} There is one \textbf{recommended} textbook for this course:
        \begin{enumerate}
            \item \textbf{International Economics}, $8^{th}$ Edition, by James Gerber (IE)
        \end{enumerate}
    It is available at the \href{https://www.uoduckstore.com/book-search-results?crn=21976&term=202402}{Duck Store (ISBN: 9780136892410)}.
    Attending lecture is not a perfect substitute for reading and comprehending the text. 
    Similarly, only reading is not a perfect substitute for for attending lecture. 
    % TALK ABOUT ARTICLES HERE
    \item \textbf{Additional Readings:} In addition to the textbook, I may assign readings from peer-reviewed studies or news articles for classroom discussion. 
    I will post any assigned additional assigned reading material on Canvas. 
    % TALK ABOUT LECTURE NOTES HERE
    \item \textbf{Lecture Notes:} Lectures will be complemented by slides. 
    They will be made available at the start of the class in which they will be presented. 
    
    \emph{Note: These serve as a guiding outline which I will follow during lectures but are not necessarily the full material. They may help keep your notes organized.}
\end{itemize}

\bigskip 

\section*{ASSIGNMENTS AND GRADING}

%\begin{table}[ht]
    \centering
    \begin{tabular}{c|l}
    \textbf{Assignment} & \textbf{Weight} \\
    \toprule
    Problem Sets $(\times 5)$  & \textbf{25 \%} \\
    Quizzes $(\times 4)$ & \textbf{10 \%} \\
    Midterm Exam  & \textbf{30 \%} \\
    Final Exam  & \textbf{30 \%} \\
    International Market Game  & \textbf{5 \%} \\
    \bottomrule
    \end{tabular}
    \label{Grade-Dist}
\end{table}

\bigskip 

\subsection*{PROBLEM SETS}
I will assign \textbf{5 Problem Sets} throughout the term. 
Problems will function as added practice for demonstrating an understanding of the course material. 
These assignments will serve as practice material for the exams. 
\begin{itemize}
    \item I will announce due dates in class within a reasonable period of time before they are due
    \item You will be turn in an \textbf{electronic copy} of each problem set on Canvas
\end{itemize}
Working in groups is encouraged as it will help you correct misunderstandings earlier on. 
Unless explicitly stated, \textbf{each student is required to write and submit their own independent answers}. 
Submitting duplicate work is subject to academic dishonesty concerns. 
If you work with others, \textbf{list their names at the top of your assignment}. 
Groups are expected to be of a reasonable size (\textbf{5 or less individuals}). 
If you fail to list collaborators, you will receive a score of zero for that problem set. 

\subsection*{QUIZZES}
I will assign \textbf{3 Quizzes} throughout the term. 
Quizzes will function as a test of your ability to use the course theory to real world contexts. 
They will be open-ended questions where you are tasked with arguing in favor or against a prompt. 
\textbf{There are no single correct answers}, but rather the main focus is on developing your use of logic/reasoning in your arguments. 

\begin{itemize}
    \item I will announce due dates in class within a reasonable period of time before they are due
    \item These will be completed on Canvas through the Quiz feature
\end{itemize}
Quizzes will be completely individual and no group work will be allowed. 
Submitting duplicate work is subject to academic dishonesty concerns. 
Answers should be no longer than one or two paragraphs. 

\subsection*{EXAMS}
All exams are taken in-person at the assigned times. 
Any form of accommodation is only valid if informed through the Accessible Education Center (AEC).
There are no make-up exams. 
Nor will there be any student able to take an exam earlier or later than the scheduled time. 
Be sure to plan Winter Break travel accordingly. 
Exams will consist of a mix of multiple-choice, short-answer, and multipart questions. 
During the exam you are allowed a set of writing utensils and a non-programmable calculator. 

\subsection*{INTERNATIONAL MARKET GAME}

Through the term we will be simulating our own global economy through some in-class activities that require your presence.   
This is meant to give you a hands-on understanding of how the theoretical concepts we will be learning appear in the world. 
You will be working in groups, behaving as a country, and producing ``goods" for sale.
We will introduce interesting circumstances and dynamics that will challenge our theory and that you will adapt to. 
These activities will take place during the second half of regular lecture days and will not be announced ahead of time. 

I will be collecting the ``data" that you generate and will keep track of things on our \href{https://jose-rojas-fallas.quarto.pub/international-econ-workbook/}{Class Workbook}.  

\bigskip \bigskip 

\section*{COURSE POLICIES \& RESOURCES}

\subsection*{ACADEMIC INTEGRITY AND HONESTY}
Academic dishonesty will not be tolerated.
This includes any form of cheating or plagiarism.
Please familiarize yourself with the \href{https://policies.uoregon.edu/vol-3-administration-student-affairs/ch-1-conduct/student-conduct-code}{Student Conduct Code}.
If there are any questions about whether an act constitutes academic misconduct, it is the students' obligation to clarify the question with the instructor before committing or attempting to commit the act.

\subsection*{ACCOMMODATIONS FOR DISABILITIES}
If you have a documented disability and anticipate needing accommodations in this course, please let me know as soon as possible.
If there are any aspects of the instruction or design of this course that result in barriers to your participation, please contact me -- your success and the success of your peers is most important. 

I encourage you to contact the \href{https://aec.uoregon.edu/}{Accessible Education Center (AEC)}. The AEC offers a wide range of support services including note-taking, testing services, sign language interpretation and adaptive technology.

\subsection*{LATE POLICY}
I will not accept late assignments after the due date. 
If you turn in a problem set or quiz on the due date but after the deadline, points will be deducted for lateness. 
If you turn in an assignment after the answer key is made public, you will receive a zero. 

I do not give makeup assignments. 
This blanket ban extends to exams. 
In extreme circumstances that lead you to miss the midterm exam, I will consider re-weighting your grade toward the final exam. 
\textbf{To qualify for re-weighting, you must notify me no later than two days after exam}. 

\subsection*{GRADE APPEALS}
You must submit any request for re-grading in writing within \textbf{three (3) business days} of the day grades are posted for a problem set or exam in question. 
Your request should include a cogent argument explaining why your response(s) warrant full credit. 

\subsection*{RESPECT FOR DIVERSITY}
You can expect to be treated with respect in this course.
Both students and the instructor enter with many identities, backgrounds, and beliefs.
Students of all racial identities, ethnicities, gender identities, gender expressions, national origins, religious affiliations, sexual orientations, immigration status, ability and other non-visible differences belong in and contribute to this class and the discipline.

The UO Economics Department welcomes and respects diverse experiences, perspectives, and approaches. 
Both nationwide and at the University of Oregon, disproportionately few women and members of historically underrepresented racial and ethnic minority groups graduate with degrees in economics. 
All class participants are expected to communicate with respect and to avoid behaviors or contributions that undermine, demean, or marginalize others based on race, ethnicity, gender, sex, age, sexual orientation, religion, ability, or socioeconomic status.

Class rosters are provided to the instructor with students' legal names.
Please let me know if the name or pronouns we are provided for yourself are not accurate.
It is important to myself and others that you are addressed in your most preferred way.

\subsection*{ACADEMIC DISRUPTION}
In the event of a campus emergency that disrupts academic activities, course requirements, deadlines, and grading percentages are subject to change. 
Information about changes in this course will be communicated as soon as possible by email and on Canvas. 
Even though we will be meeting in-person every week, students should make sure to frequently log onto Canvas and read any announcements and/or access alternative assignments. 
Students are also expected to continue coursework as outlined in this syllabus or other instructions on Canvas. 

\newpage

%\begin{table}[ht]
    \sffamily % Changes the font to correct font used in rest of text
    \centering
    \caption*{\sffamily \LARGE \textbf{TENTATIVE FALL TERM SCHEDULE}}
    \resizebox{\textwidth}{!}{%
    \begin{tabular}{l@{\hskip 0.25in} l l c}  
    \toprule \toprule
        \textbf{Week} & \textbf{Day} & \multicolumn{2}{c}{\textbf{Material}} \\
    \midrule
        \multirow{2}{*}{\textbf{Week 1}} & Mon -  (09/30) & \multicolumn{2}{c}{Ch 1}\\
        & Wed -  (10/02) & \multicolumn{2}{c}{Ch 3}\\ 
    \midrule
        \multirow{2}{*}{\textbf{Week 2}} & Mon -  (10/07) & \multicolumn{2}{c}{Ch 3}\\
        & Wed -  (10/09) & \multicolumn{2}{c}{Ch 4.1:4.3}\\ 
    \midrule
        \multirow{2}{*}{\textbf{Week 3}} & Mon -  (10/14) & \multicolumn{2}{c}{Ch 4.3:4.5}\\
        & Wed -  (10/16) & \multicolumn{2}{c}{Ch 4.6:4.7}\\ 
    \midrule
        \multirow{2}{*}{\textbf{Week 4}} & Mon -  (10/21) & \multicolumn{2}{c}{Ch 6.1}\\
        & Wed -  (10/23) & \multicolumn{2}{c}{Ch 6.2:6.3}\\ 
    \midrule
        \multirow{2}{*}{\textbf{Week 5}} & Mon -  (10/28) & \multicolumn{2}{c}{}\\
        & Wed -  (10/30) & \multicolumn{2}{c}{Reading}\\ 
    \midrule
        \multirow{2}{*}{\textbf{Week 6}} & Mon -  (11/04) & \multicolumn{2}{c}{Reading}\\
        & Wed -  (11/06) & \multicolumn{2}{c}{Reading}\\ 
    \midrule
        \multirow{2}{*}{\textbf{Week 7}} & Mon -  (11/11) & \multicolumn{2}{c}{Reading}\\
        & Wed -  (11/13) & \multicolumn{2}{c}{Reading}\\ 
    \midrule
        \multirow{2}{*}{\textbf{Week 8}} & Mon -  (11/18) & \multicolumn{2}{c}{Reading}\\
        & Wed -  (11/20) & \multicolumn{2}{c}{Reading}\\ 
    \midrule
        \multirow{2}{*}{\textbf{Week 9}} & Mon -  (11/25) & \multicolumn{2}{c}{Reading}\\
        & Wed -  (11/27) & \multicolumn{2}{c}{Reading}\\ 
    \midrule
        \multirow{2}{*}{\textbf{Week 10}} & Mon -  (12/02) & \multicolumn{2}{c}{Reading}\\
        & Wed -  (12/04) & \multicolumn{2}{c}{Reading}\\ 
    \midrule
        \textbf{Finals Week} & Day & \multicolumn{2}{c}{\textbf{FINAL}} \\
    \bottomrule \bottomrule
    \multicolumn{4}{c}{\footnotesize Topics are subject to change depending on class pace. The content will not. } \\
    \multicolumn{4}{c}{\footnotesize I will update dates as needed during the term.}
    \end{tabular}
    }
    \label{tab:schedule}
\end{table}

\end{document}