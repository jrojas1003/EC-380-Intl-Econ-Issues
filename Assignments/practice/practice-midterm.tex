\documentclass[12pt]{exam}
\usepackage{amsthm}
\usepackage{libertine}
\usepackage[utf8]{inputenc}
\usepackage[margin=1in]{geometry}
\usepackage{amsmath,amssymb}
\usepackage{multicol}
\usepackage[shortlabels]{enumitem}
\usepackage{siunitx}
\usepackage{cancel}
\usepackage{graphicx}
\usepackage{listings}
\usepackage{tikz}
\usepackage{soul}

%\printanswers % Comment out to hide answers 
\CorrectChoiceEmphasis{\color{red}\bfseries\itshape}

\newcommand{\class}{\textbf{EC 380}} % This is the name of the course 
\newcommand{\examnum}{\textbf{Winter 25 - Practice Midterm}} % This is the name of the assignment
\newcommand{\examdate}{\textbf{}} % This is the due date
\newcommand{\timelimit}{}
\addpoints % Allows to add points up to include table at end of document

\pagestyle{headandfoot} % Fancy equivalent for exam documentclass
\firstpageheadrule % Horizontal bar in first page
\runningheadrule % Horizontal bar in rest of pages
\firstpageheader{\class}{\examnum}{\examdate} % 1st page header content
\firstpagefooter{Points earned: \makebox[1in]{\hrulefill} / \pointsonpage{\thepage} points}{}{\thepage\ of \numpages} % 1st page footer content
\runningheader{\class}{\examnum}{\examdate} % Rest of pages header content
\runningfooter{Points earned: \makebox[1in]{\hrulefill} / \pointsonpage{\thepage} points}{}{\thepage\ of \numpages} % Rest of pages footer content
\bracketedpoints % Puts points per question inside brackets instead of parenthesis 


\title{Practice Midterm Exam}
\author{EC 380 - International Economic Issues}
\date{Winter 2025}

\begin{document}

%-------------------------------------------------------------------------------
\begin{coverpages}

\maketitle

\begin{center}
\fbox{\fbox{\parbox{5.5in}{\centering
Attempt all problems. \\
At the very least, identify where you no longer know what to do. \\
Be aware of timing. \\
}}}\end{center}
\vspace{2cm}
\makebox[0.6\textwidth]{Name:\enspace\hrulefill}
\makebox[0.35\textwidth]{95\#:\enspace\hrulefill}

\vspace{1cm}
\noindent
The maximum amount of points on this exam is \hl{X points}. 
You have a total of 1h 20min (80 minutes) to complete the exam, unless otherwise noted. 
The only items allowed on your desk at any time are a pen and/or pencil, scratch paper, a 3x5 note card, and a calculator. 
Everything else must be stored in your bag underneath your desk. 
Any form of cheating will result on a zero on the exam.\\

\noindent There are three sections to be completed:

\begin{itemize}
    \item \textbf{Multiple Choice:} 3 Questions
    \item \textbf{Short Answer Questions:} 1 Questions
    \item \textbf{Multi-Part Analysis Questions:} 1 Question
\end{itemize}

\noindent Point totals and question specific instructions are listed for each section.
Please ask for clarification if a question is not clear to you.\\

\noindent The exam is a total of \numpages $\,$ pages. 
\textbf{Please verify you have all \numpages $\,$ in your exam. If you do not, let me know immediately.}

\noindent 

\end{coverpages}
%-------------------------------------------------------------------------------

%-------------------------------------------------------------------------------
\section*{Multiple Choice - 20 Points}
Circle or "X" the answer you think most correctly answers the following questions. 
If you mark a choice and would like to change it, \textbf{clearly indicate which one is your correct answer}. 
%-------------------------------------------------------------------------------
\begin{questions}

\question[4] % Question 1
Under \textbf{autarky}, what must the relative price of a good be equal to?

    \begin{choices}
        \choice The cost of inputs
        \choice The world price
        \correctchoice Its opportunity cost of production
        \choice The consumer's willingness to pay
    \end{choices}
\vspace*{\stretch{1}}

\question[4] % Question 3
The \textbf{Stolper-Samuelson theorem} states that a \textbf{decrease} in the price of a good \fillin[][0.5in] the income earned by factors are that are used \fillin[][0.5in] in its production
    \begin{choices}
        \correctchoice Lowers; intensively
        \choice Raises; intensively
        \choice Lowers; non-intensively
        \choice Raises; non-intensively
    \end{choices}
\vspace*{\stretch{1}}

\question[4] % Question 5
When trade occurs, we expect the price of the \textbf{import good} to be \fillin[][0.5in] relative to pre-trade prices. 
    \begin{choices}
        \choice Equal
        \correctchoice Lower
        \choice Higher
        \choice Cannot tell without knowing more
    \end{choices}
\vspace*{\stretch{1}}
%-------------------------------------------------------------------------------
\newpage
\section*{Short Answer - 15 Points}
Answer the following questions to the best of your ability.
For full credit, show all of your work and clearly indicate your final solution for each party by circling the answer.
%-------------------------------------------------------------------------------

\question
Suppose that Mexico and Costa Rica have the following capital and labor endowments:
\begin{table}[ht]
    \centering
    %\caption{Countries Productivity Structure}
    \begin{tabular}{|c|c|c|}
    \hline
     & Capital & Labor \\
    \hline
    Mexico & 540 & 650 \\
    \hline
    Costa Rica & 455 & 480 \\
    \hline
    \end{tabular}
    \end{table}

\begin{parts}
    \part[5] 
    Which country has the comparative advantage in producing the \textbf{labor-intensive good}?
    \vspace*{\stretch{1}}
    \part[5] 
    How would things change in terms of comparative advantages if suddenly migrants entered Mexico, causing labor to jump up by an additional 200 units?
    \vspace*{\stretch{1}}
\end{parts}

%-------------------------------------------------------------------------------
\newpage
\section*{Multi-Part Analysis - 30 Points}
Answer the following questions to the best of your ability.
For full credit, show all of your work and clearly indicate your final solution for each party by circling the answer.
%-------------------------------------------------------------------------------

%-------------------------------------------------------------------------------
\question
Suppose we are in an \textbf{autarky scenario} and considering the market for an imported good at Home. 
Use the following \textbf{demand and supply} functions for solving the various equilibrium scenarios.

\begin{align*}
    \text{Demand: }& P = 415 - \dfrac{1}{7}Q_{d} \\
    \text{Supply: }& P = 10 + \dfrac{1}{4}Q_{s}
\end{align*}

\textbf{Be sure to show all your work}

\begin{parts}
    \part[5] 
    Report the coordinates of the equilibrium point, which represent the \textbf{price and quantity the market operates at}
    \vspace*{\stretch{1}}
    \part[5] 
    Calculate the consumer and producer surplus values under \textbf{autarky}. 
    What is the total welfare for the economy?
    \vspace*{\stretch{1}}
    \newpage
    \part[6]
    Suppose the government imposes a tariff of $t$ value on every good. 
    What would we \textbf{expect to happen to quantity supplied, quantity demanded, and price, relative to free-trade}?
    Provide a clear answer for each part mentioned and briefly explain why you expect those movements.
    \vspace*{2cm}
        \begin{parts}
            \part \textbf{Quantity Supplied}
            \vspace*{\stretch{1}}
            \part \textbf{Quantity Demanded}
            \vspace*{\stretch{1}}
            \part \textbf{Price}
            \vspace*{\stretch{1}}
        \end{parts}
\end{parts}

\end{questions}
%-------------------------------------------------------------------------------
\end{document}