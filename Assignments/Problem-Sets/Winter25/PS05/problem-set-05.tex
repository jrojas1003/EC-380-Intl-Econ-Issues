\documentclass[12pt]{exam}
\usepackage{amsthm}
\usepackage{libertine}
\usepackage[utf8]{inputenc}
\usepackage[margin=1in]{geometry}
\usepackage{amsmath,amssymb}
\usepackage{multicol}
\usepackage[shortlabels]{enumitem}
\usepackage{siunitx}
\usepackage{cancel}
\usepackage{graphicx}
\usepackage{listings}
\usepackage{tikz}
\usepackage[T1]{fontenc}  % Ensure proper encoding
\usepackage{titlesec}
\usepackage{booktabs}

\CorrectChoiceEmphasis{\color{red}\bfseries\itshape}

\newcommand{\class}{\fontfamily{lmss}\selectfont \textbf{EC 380}} % This is the name of the course 
\newcommand{\examnum}{\fontfamily{lmss}\selectfont \textbf{}} % This is the name of the assignment
\newcommand{\examdate}{\fontfamily{lmss}\selectfont \textbf{March 05 at 01:59pm}} % This is the due date
\newcommand{\timelimit}{}
\addpoints % Allows to add points up to include table at end of document

\pagestyle{headandfoot} % Fancy equivalent for exam documentclass
\firstpageheadrule % Horizontal bar in first page
\runningheadrule % Horizontal bar in rest of pages
\firstpageheader{\class}{\examnum}{\fontfamily{lmss}\selectfont  \examdate} % 1st page header content
\firstpagefooter{}{Page \thepage}{} % 1st page footer content
\runningheader{\class}{\examnum}{\fontfamily{lmss}\selectfont  \examdate} % Rest of pages header content
%\runningfooter{Points earned: \makebox[1in]{\hrulefill} / \pointsonpage{\thepage} points}{}{\thepage\ of \numpages} % Rest of pages footer content
\bracketedpoints % Puts points per question inside brackets instead of parenthesis 

\titleformat{\section}
  {\normalfont\large\bfseries\sffamily}{\thesection}{1em}{}

\titleformat{\subsection}
  {\normalfont\normalsize\bfseries\sffamily}{\thesection}{1em}{}

\begin{document}
\fontfamily{lmss}\selectfont

%%%%%%%%%%%%%%%%%%%%%%%%%%%%%%%%%%%%%%% PREAMBLE %%%%%%%%%%%%%%%%%%%%%%%%%%%%%%%%%%%%%%%
\begin{center}
    \textbf{{\LARGE Problem Set 05}} \\
    \bigskip 
\end{center}

\noindent \textbf{Instructions:} 
Answers must be submitted online through the designated Canvas assignment. 
This Problem Set is due on \examdate.
Please write as legible and clearly as possible. 
You will not be given full credit if your answers cannot be easily understood. 

%%%%%%%%%%%%%%%%%%%%%%%%%%%%%%%%%%%%%%%%%%%%%%%%%%%%%%%%%%%%%%%%%%%%%%%%%%%%%%%%%%%%%%%%

%%%%%%%%%%%%%%%%%%%%%%%%%%%%%%%%%%%%%%% QUESTIONS %%%%%%%%%%%%%%%%%%%%%%%%%%%%%%%%%%%%%%
\section*{Questions}
\begin{questions}

\question 
Consider the following demand and supply curves for foreign currency, where \textbf{ExR} represents the exchange rate of local currency to foreign currency (e.g. USD: GBP).
\textbf{FC} represents the units of foreign currency reserves held in the ``local" economy.

\begin{align*}
    \text{Demand}: \text{ExR} = 3 - 0.075FC \;\;\; ; \;\;\; \text{Supply}: \text{ExR} = 0.5 + 0.025FC      
\end{align*}

\begin{parts}
    \part What is the \textbf{Exchange Rate} and \textbf{Foreign Currency Reserves} amount?
    \vspace*{\stretch{2}}
    \part Consider a case in which the foreign interest rate, $i^{*}$, rises such that demand sees a \textbf{0.8 level-shift increase} and the new demand curve can be represented by $D^{'} = D + 0.8$. 
    What are the new exchange rate and currency reserve values?
    \vspace*{\stretch{2}}
    \part How would you describe the change in \textbf{both currencies}?
    Which has depreciated and which has appreciated?
    \vspace*{\stretch{1}}
\end{parts}

\newpage

\question 
Consider purchasing power parity (PPP) holding across long-run exchange rates. 
Suppose an identical basket of goods is available in the US and Japan. 
In the US, the goods are valued at 1,400 USD whereas in Japan they are valued at 194,775 Japanese Yen (JPY). 

\begin{parts}
    \part What is the implicit USD-JPY exchange rate, if the PPP relationship is satisfied?
    \vspace*{\stretch{1}}
    \part Suppose that the exchange rate, USD-JPY, is currently 112 JPY per USD. 
    Is the UDS undervalued or overvalued?
    Explain why. 
    \vspace*{\stretch{1}}
    \part How would a Japanese merchant go about exploiting price differences between the US and Japan for the same basket of identical goods?
    \vspace*{\stretch{1}}
    \part What effect would these actions have on the USD-JPY exchange rate?
    When would price pressures on the exchange rate stop?
    \vspace*{\stretch{1}}
\end{parts}

\end{questions}

\end{document}