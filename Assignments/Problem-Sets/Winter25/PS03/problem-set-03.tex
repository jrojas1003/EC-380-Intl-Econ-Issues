\documentclass[12pt]{exam}
\usepackage{amsthm}
\usepackage{libertine}
\usepackage[utf8]{inputenc}
\usepackage[margin=1in]{geometry}
\usepackage{amsmath,amssymb}
\usepackage{multicol}
\usepackage[shortlabels]{enumitem}
\usepackage{siunitx}
\usepackage{cancel}
\usepackage{graphicx}
\usepackage{listings}
\usepackage{tikz}
\usepackage[T1]{fontenc}  % Ensure proper encoding
\usepackage{titlesec}
\usepackage{booktabs}

\CorrectChoiceEmphasis{\color{red}\bfseries\itshape}

\newcommand{\class}{\fontfamily{lmss} \textbf{EC 380}} % This is the name of the course 
\newcommand{\examnum}{\fontfamily{lmss} \textbf{PS 03}} % This is the name of the assignment
\newcommand{\examdate}{\fontfamily{lmss} \textbf{January 29 at 01:59pm}} % This is the due date
\newcommand{\timelimit}{}
\addpoints % Allows to add points up to include table at end of document

\pagestyle{headandfoot} % Fancy equivalent for exam documentclass
\firstpageheadrule % Horizontal bar in first page
\runningheadrule % Horizontal bar in rest of pages
\firstpageheader{\class}{\examnum}{\fontfamily{lmss} Due \examdate} % 1st page header content
\firstpagefooter{\fontfamily{lmss} Points earned: \makebox[1in]{\hrulefill} / \pointsonpage{\thepage} points}{}{\thepage\ of \numpages} % 1st page footer content
\runningheader{\class}{\examnum}{\fontfamily{lmss} Due \examdate} % Rest of pages header content
\runningfooter{\fontfamily{lmss} Points earned: \makebox[1in]{\hrulefill} / \pointsonpage{\thepage} points}{}{\thepage\ of \numpages} % Rest of pages footer content
\bracketedpoints % Puts points per question inside brackets instead of parenthesis 

\titleformat{\section}
  {\normalfont\large\bfseries\sffamily}{\thesection}{1em}{}

\titleformat{\subsection}
  {\normalfont\normalsize\bfseries\sffamily}{\thesection}{1em}{}

\begin{document}
\fontfamily{lmss}
\selectfont

%%%%%%%%%%%%%%%%%%%%%%%%%%%%%%%%%%%%%%% PREAMBLE %%%%%%%%%%%%%%%%%%%%%%%%%%%%%%%%%%%%%%%
\begin{center}
    \textbf{{\LARGE Problem Set 03}} \\
    \bigskip 
\end{center}

\noindent \textbf{Instructions:} 
Answers must be submitted online through the designated Canvas assignment. 
This Problem Set is due on \examdate.
Please write as legible and clearly as possible. 
You will not be given full credit if your answers cannot be easily understood. 

%%%%%%%%%%%%%%%%%%%%%%%%%%%%%%%%%%%%%%%%%%%%%%%%%%%%%%%%%%%%%%%%%%%%%%%%%%%%%%%%%%%%%%%%

%%%%%%%%%%%%%%%%%%%%%%%%%%%%%%%%%%%%%%% QUESTIONS %%%%%%%%%%%%%%%%%%%%%%%%%%%%%%%%%%%%%%
\section*{Questions}
\begin{questions}

\question
Suppose we are in an autarky scenario and considering the market for an imported good at Home. 
Use the following demand and supply functions for solving the various equilibrium scenarios:
\begin{align*}
    \textbf{Demand:} \;\; P &= 120 - \dfrac{4}{7}Q_{d} \\
    \textbf{Supply:} \;\; P &= \dfrac{1}{4}Q_{s}
\end{align*}

Consider the \textbf{Autarky Scenario} first

\begin{parts}
    \part[3]  
    Sketch the supply and demand curves, with the appropriate labeling for the equilibrium point and surplus regions.
    \vspace*{\stretch{1}}
    \part[5] Report the coordinates of the equilibrium point, which represent the \textbf{price and quantity the market operates at}.
    \vspace*{\stretch{2}} 
    \part[3] Calculate the consumer and producer surplus values under autarky. 
    What is the total welfare for the economy?
    \vspace*{\stretch{0.75}} 
\end{parts}

\newpage 

\question
Using the same demand and supply functions as before, answer the following:
\begin{align*}
    \textbf{Demand:} \;\; P &= 120 - \dfrac{4}{7}Q_{d} \\
    \textbf{Supply:} \;\; P &= \dfrac{1}{4}Q_{s}
\end{align*}

Suppose Home opens up to \textbf{free-trade} and becomes exposed to a world price, $P_{w} = 25$.
Be sure to complete every part.
\begin{parts}
    \part[3]  
    Sketch the market with the \textbf{new price line} and corresponding equilibria points for \textbf{quantity demanded and supplied}.
    \vspace*{\stretch{1.5}}
    \part[5]
    Calculate the equilibrium values for quantities, imports, and surplus values. 
    \vspace*{\stretch{2}}
    \part[3]
    What is the change in welfare, relative to \textbf{autarky}
    \vspace*{\stretch{0.75}}
\end{parts}

\newpage 

\question
Using the same demand and supply functions as before, answer the following:
\begin{align*}
    \textbf{Demand:} \;\; P &= 120 - \dfrac{4}{7}Q_{d} \\
    \textbf{Supply:} \;\; P &= \dfrac{1}{4}Q_{s}
\end{align*}

Consider the case in which \textbf{the government intervenes, setting a tariff rate of $t = 4$}.
Be sure to complete every part.
\begin{parts}
    \part[3]
    Sketch the updated demand \& supply curves. 
    Label it properly and highlight which regions are the efficiency and dead-weight loss areas
    \vspace*{\stretch{1.5}}
    \part[5] 
    Calculate the equilibria for \textbf{quantity supplied, quantity demanded, imports, and surpluses (consumer, producer, government)}.
    \vspace*{\stretch{2}}
    \part[3] 
    What is the \textbf{change in welfare}, relative to free-trade?
    \vspace*{\stretch{0.75}}
\end{parts}

\newpage 

\begin{table}[ht]
    \centering
    \begin{tabular}{lc|c|c}
    \toprule
     Variable & No Tariff & + Tariff on Final Good & + Tariff on Input Good \\
    \midrule
     Price of Domestic Final Good & 2220 & & \\
     \midrule
     Value of Imported Inputs & 670 & & \\
     \midrule
     Domestic Value-Added & 1550 & & \\
    \midrule
     Effective Rate of Protection, \% & 0 & & \\
     \bottomrule 
    \end{tabular}
    \label{tab:my_label}
\end{table}

\question[10]
Complete the table above and express the effective rate of protection in \textbf{each case}. 
\textbf{Tariffs on the final good are 25\% and tariffs on the input good are 12\%}.
\textbf{Show your work in the space provided below}
\vspace*{\stretch{1}}


\end{questions}

\end{document}
