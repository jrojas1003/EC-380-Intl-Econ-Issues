\documentclass[12pt]{exam}
\usepackage{amsthm}
\usepackage{libertine}
\usepackage[utf8]{inputenc}
\usepackage[margin=1in]{geometry}
\usepackage{amsmath,amssymb}
\usepackage{multicol}
\usepackage[shortlabels]{enumitem}
\usepackage{siunitx}
\usepackage{cancel}
\usepackage{graphicx}
\usepackage{listings}
\usepackage{tikz}
\usepackage[T1]{fontenc}  % Ensure proper encoding
\usepackage{titlesec}
\usepackage{booktabs}

\CorrectChoiceEmphasis{\color{red}\bfseries\itshape}

\newcommand{\class}{\fontfamily{lmss} \textbf{EC 380}} % This is the name of the course 
\newcommand{\examnum}{\fontfamily{lmss} \textbf{PS 02}} % This is the name of the assignment
\newcommand{\examdate}{\fontfamily{lmss} \textbf{January 22 at 01:59pm}} % This is the due date
\newcommand{\timelimit}{}
\addpoints % Allows to add points up to include table at end of document

\pagestyle{headandfoot} % Fancy equivalent for exam documentclass
\firstpageheadrule % Horizontal bar in first page
\runningheadrule % Horizontal bar in rest of pages
\firstpageheader{\class}{\examnum}{\fontfamily{lmss} Due \examdate} % 1st page header content
\firstpagefooter{\fontfamily{lmss} Points earned: \makebox[1in]{\hrulefill} / \pointsonpage{\thepage} points}{}{\thepage\ of \numpages} % 1st page footer content
\runningheader{\class}{\examnum}{\fontfamily{lmss} Due \examdate} % Rest of pages header content
\runningfooter{\fontfamily{lmss} Points earned: \makebox[1in]{\hrulefill} / \pointsonpage{\thepage} points}{}{\thepage\ of \numpages} % Rest of pages footer content
\bracketedpoints % Puts points per question inside brackets instead of parenthesis 

\titleformat{\section}
  {\normalfont\large\bfseries\sffamily}{\thesection}{1em}{}

\titleformat{\subsection}
  {\normalfont\normalsize\bfseries\sffamily}{\thesection}{1em}{}

\begin{document}
\fontfamily{lmss}
\selectfont

%%%%%%%%%%%%%%%%%%%%%%%%%%%%%%%%%%%%%%% PREAMBLE %%%%%%%%%%%%%%%%%%%%%%%%%%%%%%%%%%%%%%%
\begin{center}
    \textbf{{\LARGE EC 380 Problem Set 02}} \\
    \bigskip 
\end{center}

\noindent \textbf{Instructions:} 
Answers must be submitted online through the designated Canvas assignment in a \textbf{PDF file}.
Any other file type is not allowed.
This Problem Set is due on \examdate.
Please write as legible and clearly as possible. 
You will not be given full credit if your answers cannot be easily understood. 

%%%%%%%%%%%%%%%%%%%%%%%%%%%%%%%%%%%%%%%%%%%%%%%%%%%%%%%%%%%%%%%%%%%%%%%%%%%%%%%%%%%%%%%%

%%%%%%%%%%%%%%%%%%%%%%%%%%%%%%%%%%%%%%% QUESTIONS %%%%%%%%%%%%%%%%%%%%%%%%%%%%%%%%%%%%%%
\section*{Questions}
\begin{questions}

\question
Answer the following short questions

\begin{parts}
    \part[4] In your own words, how would you define \textbf{Labor Abundance} in the Heckscher-Ohlin model setting?
    \vspace*{\stretch{1}}
    \part[4] How does the capital-labor ratio help us determine patterns of trade?
    \vspace*{\stretch{1}}
    \part[4] Describe the key difference(s) that separate the HO model from the Ricardian model
    \vspace*{\stretch{1}}
\end{parts}
%\vspace*{\stretch{1}}

\newpage 

\question
Suppose we are considering an \textbf{HO Model setting}, where countries have not yet opened up to trade.
Two goods are produced: \textbf{Suits and Ozempic}. 
Suppose that tailoring \textbf{Suits} is \textbf{labor-intensive} in production as it is primarily done by hand and making \textbf{Ozempic} is \textbf{capital-intensive} in production as it requires exact and automated machinery. 
The countries, Country A and Country B, have the following \textbf{Labor (L) and Capital (K)} endowments.

\begin{table}[ht]
    \centering
    %\caption{Countries Productivity Structure}
    \begin{tabular}{|c|c|c|}
    \hline
     & Rubber Ducks & Bath Bombs \\
    \hline
    Home & 20 & 18 \\
    \hline
    Foreign & 14 & 8 \\
    \hline
    \end{tabular}
\end{table}

\begin{parts}
    \part[4] What are the \textbf{Capital-Labor Ratios} for each country?
    \vspace*{\stretch{1}}
    \part[4] Which country has comparative advantage in producing Ozempic?
    \vspace*{\stretch{1}} 
    \part[4] How will trade flows look once each country specializes?
    \vspace*{\stretch{1}}
\end{parts}

\newpage 

\question[8]
How do trade outcomes play out for owners of factor inputs when trade opens? 
Address both capital and labor outcome when a country is abundant in either capital or labor.
\vspace*{\stretch{1}}

\question[6]
Consider the \textbf{Specific-Factor Model}. 
What are the implications for domestic labor income of switching from autarky to open trade, when a given country is \textbf{land-abundant}?
Is this impact different if the country is \textbf{capital-abundant} instead?
\vspace*{\stretch{1}}

\newpage 

\question[8]
Recall the \textbf{Magnification Effect} we learned in lecture.
Let the capital share be 27\% and the labor share be 73\% to produce Banana Bread.
If the rental rate of capital were to remain constant, then a 12\% increase in the price of Banana Bread must be accompanied by what percentage increase/decrease in wages?
\vspace*{\stretch{1}}

\question[4]
We covered ambiguities of changes in labor outcomes, given greater openness to trade. 
Who does \textbf{Autor et al (2013)} suggest are the most vulnerable with respect to labor market outcomes in the US, following Chinese trade liberalization between 1990 and 2007?
\vspace*{\stretch{1}}

\end{questions}

\end{document}
