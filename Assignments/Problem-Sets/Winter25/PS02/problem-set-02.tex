\documentclass[12pt]{exam}
\usepackage{amsthm}
\usepackage{libertine}
\usepackage[utf8]{inputenc}
\usepackage[margin=1in]{geometry}
\usepackage{amsmath,amssymb}
\usepackage{multicol}
\usepackage[shortlabels]{enumitem}
\usepackage{siunitx}
\usepackage{cancel}
\usepackage{graphicx}
\usepackage{listings}
\usepackage{tikz}
\usepackage[T1]{fontenc}  % Ensure proper encoding
\usepackage{titlesec}
\usepackage{booktabs}

\CorrectChoiceEmphasis{\color{red}\bfseries\itshape}

\newcommand{\class}{\fontfamily{lmss} \textbf{EC 380}} % This is the name of the course 
\newcommand{\examnum}{\fontfamily{lmss} \textbf{PS 02}} % This is the name of the assignment
\newcommand{\examdate}{\fontfamily{lmss} \textbf{January 22 at 01:59pm}} % This is the due date
\newcommand{\timelimit}{}
\addpoints % Allows to add points up to include table at end of document

\pagestyle{headandfoot} % Fancy equivalent for exam documentclass
\firstpageheadrule % Horizontal bar in first page
\runningheadrule % Horizontal bar in rest of pages
\firstpageheader{\class}{\examnum}{\fontfamily{lmss} Due \examdate} % 1st page header content
\firstpagefooter{\fontfamily{lmss} Points earned: \makebox[1in]{\hrulefill} / \pointsonpage{\thepage} points}{}{\thepage\ of \numpages} % 1st page footer content
\runningheader{\class}{\examnum}{\fontfamily{lmss} Due \examdate} % Rest of pages header content
\runningfooter{\fontfamily{lmss} Points earned: \makebox[1in]{\hrulefill} / \pointsonpage{\thepage} points}{}{\thepage\ of \numpages} % Rest of pages footer content
\bracketedpoints % Puts points per question inside brackets instead of parenthesis 

\titleformat{\section}
  {\normalfont\large\bfseries\sffamily}{\thesection}{1em}{}

\titleformat{\subsection}
  {\normalfont\normalsize\bfseries\sffamily}{\thesection}{1em}{}

\begin{document}
\fontfamily{lmss}
\selectfont

%%%%%%%%%%%%%%%%%%%%%%%%%%%%%%%%%%%%%%% PREAMBLE %%%%%%%%%%%%%%%%%%%%%%%%%%%%%%%%%%%%%%%
\begin{center}
    \textbf{{\LARGE EC 380 Problem Set 02}} \\
    \bigskip 
\end{center}

\noindent \textbf{Instructions:} 
Answers must be submitted online through the designated Canvas assignment in a \textbf{PDF file}.
Any other file type is not allowed.
This Problem Set is due on \examdate.
Please write as legible and clearly as possible. 
You will not be given full credit if your answers cannot be easily understood. 

%%%%%%%%%%%%%%%%%%%%%%%%%%%%%%%%%%%%%%%%%%%%%%%%%%%%%%%%%%%%%%%%%%%%%%%%%%%%%%%%%%%%%%%%

%%%%%%%%%%%%%%%%%%%%%%%%%%%%%%%%%%%%%%% QUESTIONS %%%%%%%%%%%%%%%%%%%%%%%%%%%%%%%%%%%%%%
\section*{Questions}
\begin{questions}

\question
Answer the following short questions

\begin{parts}
    \part[4] In your own words, how would you define \textbf{Labor Abundance} in the Heckscher-Ohlin model setting?
    \vspace*{\stretch{1}}
    \part[4] How does the capital-labor ratio help us determine patterns of trade?
    \vspace*{\stretch{1}}
    \part[4] Describe the key difference(s) that separate the HO model from the Ricardian model
    \vspace*{\stretch{1}}
\end{parts}
%\vspace*{\stretch{1}}

\newpage 

\question
Suppose we are considering an \textbf{HO Model setting}, where countries have not yet opened up to trade.
Two goods are produced: \textbf{Suits and Distilled Whiskey}. 
Suppose that tailoring \textbf{Suits} is \textbf{labor-intensive} in production as it is primarily done by hand and making \textbf{Distilled Whiskey} is \textbf{capital-intensive} in production as it requires exact and automated machinery. 
The countries, Country A and Country B, have the following \textbf{Labor (L) and Capital (K)} endowments.

\begin{table}[ht]
    \centering
    %\caption{Countries Productivity Structure}
    \begin{tabular}{|c|c|c|}
    \hline
     & Rubber Ducks & Bath Bombs \\
    \hline
    Home & 20 & 18 \\
    \hline
    Foreign & 14 & 8 \\
    \hline
    \end{tabular}
\end{table}

\begin{parts}
    \part[4] What are the \textbf{Capital-Labor Ratios} for each country?
    \vspace*{\stretch{1}}
    \part[4] Which country has comparative advantage in producing Distilled Whiskey?
    \vspace*{\stretch{1}} 
    \part[4] How do trade flows behave for each country once each of them specializes?
    \vspace*{\stretch{1}}
\end{parts}

\newpage 

\question
Consider the gains and losses experienced by owners of input factors. 
Assume that in this economy there is only \textbf{Capital and Labor}. 

\begin{parts}
  \part[4] How does trade affect owners of capital and owners of labor under a \textbf{Capital Abundant country}?
  \vspace*{\stretch{1}}
  \part[4] How does trade affect owners of capital and owners of labor under a \textbf{Labor Abundant country}?
  \vspace*{\stretch{1}}
\end{parts}

\question
Consider the \textbf{Specific-Factor Model}.
Assume that \textbf{Land} and \textbf{Capital} are the \textbf{Specific Factors}.

\begin{parts}
  \part[4] What are the implications for domestic labor income of switching from autarky to open trade, when a given country is \textbf{land-abundant}?
  \vspace*{\stretch{1}}
  \part[2] Is this impact different if the country is \textbf{capital-abundant} instead?
  \vspace*{\stretch{1}}
\end{parts}

\newpage 

\question[12] 
Consider a small, open economy that produces two goods: \textbf{Exotic Flowers} and \textbf{Semi-conductors}.
In order to make either good, producers must use both factors of production: \textbf{Labor} and \textbf{Capital}.
Either good requires a specific mix of input factors:

\begin{multicols}{2}
  \centering 
  \textbf{Exotic Flowers}
  \begin{itemize}
    \item 64\% Labor \& 36\% Capital 
  \end{itemize}
  
  \columnbreak
  
  \centering
  \textbf{Semi-conductors}
  \begin{itemize}
    \item 23\% Labor \& 77\% Capital
  \end{itemize}
  \end{multicols}

The economy initially operates under \textbf{autarky}. 
After opening to trade, the world relative price of \textbf{Exotic Flowers} increases from 10 to 12. 
Assume the economy adjusts fully to the new price ratios, with full employment. 

\textbf{Hint:} 
The percentage change in the price of a good is given by:
\begin{align*}
  \Delta P_{x} = \dfrac{P_{x}^{new} - P_{x}^{old}}{P_{x}^{old}} \times 100
\end{align*}

Using the \textbf{Magnification Effect} we learned in lecture, find the percentage change in \textbf{wages $(\Delta w)$} for labor and the percentage change in \textbf{return to capital $(\Delta r)$} for capital

\end{questions}

\end{document}
