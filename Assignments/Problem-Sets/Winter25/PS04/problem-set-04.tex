\documentclass[12pt]{exam}
\usepackage{amsthm}
\usepackage{libertine}
\usepackage[utf8]{inputenc}
\usepackage[margin=1in]{geometry}
\usepackage{amsmath,amssymb}
\usepackage{multicol}
\usepackage[shortlabels]{enumitem}
\usepackage{siunitx}
\usepackage{cancel}
\usepackage{graphicx}
\usepackage{listings}
\usepackage{tikz}
\usepackage[T1]{fontenc}  % Ensure proper encoding
\usepackage{titlesec}
\usepackage{booktabs}

\CorrectChoiceEmphasis{\color{red}\bfseries\itshape}

\newcommand{\class}{\fontfamily{lmss}\selectfont \textbf{EC 380}} % This is the name of the course 
\newcommand{\examnum}{\fontfamily{lmss}\selectfont \textbf{}} % This is the name of the assignment
\newcommand{\examdate}{\fontfamily{lmss}\selectfont \textbf{February 26 at 01:59pm}} % This is the due date
\newcommand{\timelimit}{}
\addpoints % Allows to add points up to include table at end of document

\pagestyle{headandfoot} % Fancy equivalent for exam documentclass
\firstpageheadrule % Horizontal bar in first page
\runningheadrule % Horizontal bar in rest of pages
\firstpageheader{\class}{\examnum}{\fontfamily{lmss}\selectfont Due \examdate} % 1st page header content
\firstpagefooter{}{Page \thepage}{} % 1st page footer content
\runningheader{\class}{\examnum}{\fontfamily{lmss}\selectfont Due \examdate} % Rest of pages header content
%\runningfooter{Points earned: \makebox[1in]{\hrulefill} / \pointsonpage{\thepage} points}{}{\thepage\ of \numpages} % Rest of pages footer content
\bracketedpoints % Puts points per question inside brackets instead of parenthesis 

\titleformat{\section}
  {\normalfont\large\bfseries\sffamily}{\thesection}{1em}{}

\titleformat{\subsection}
  {\normalfont\normalsize\bfseries\sffamily}{\thesection}{1em}{}

\begin{document}
\fontfamily{lmss}\selectfont

%%%%%%%%%%%%%%%%%%%%%%%%%%%%%%%%%%%%%%% PREAMBLE %%%%%%%%%%%%%%%%%%%%%%%%%%%%%%%%%%%%%%%
\begin{center}
    \textbf{{\LARGE Problem Set 04}} \\
    \bigskip 
\end{center}

\noindent \textbf{Instructions:} 
Answers must be submitted online through the designated Canvas assignment. 
This Problem Set is due on \examdate.
Please write as legible and clearly as possible. 
You will not be given full credit if your answers cannot be easily understood. 

%%%%%%%%%%%%%%%%%%%%%%%%%%%%%%%%%%%%%%%%%%%%%%%%%%%%%%%%%%%%%%%%%%%%%%%%%%%%%%%%%%%%%%%%

%%%%%%%%%%%%%%%%%%%%%%%%%%%%%%%%%%%%%%% QUESTIONS %%%%%%%%%%%%%%%%%%%%%%%%%%%%%%%%%%%%%%
\section*{Questions}
\begin{questions}


\question[15]
Answer the following short questions:

\begin{parts}
    \part[5] What type of activities does the \textbf{capital account} consist of?
    \vspace*{\stretch{1}}
    \part[5] Hoes do the \textbf{current, capital, and financial accounts} relate to one another when it comes to their numeric value?
    \vspace*{\stretch{1}}
    \part[5] Explain the reason why the balance of payments features a \textbf{statistical discrepancy value}.
    \vspace*{\stretch{1}}
\end{parts}

\newpage 

\question[25]
Consider the following balance of payments for a given country 

\begin{table}[ht]
    \centering
    \begin{tabular}{l|lc}
    \toprule
     \textbf{ID} & \textbf{Description} & \textbf{Billions (USD)} \\
     \midrule
     1. & Current Account Balance & 1000 \\
     2. & Capital Account Balance & 200 \\
     3. & Financial Account & - \\
     3.a & Net Acq. of Financial Assets, Excl. Financial Der. (Increase/Outflow (+)) & 470 \\
     3.b & Net Inc. of Liabilities, Excl. Financial Der. (Increase/Inflow (+)) & -890 \\
     3.c & Net Change in Financial Derivatives & -200 \\
     4. & Statistical Discrepancy & \\
     5. & Memoranda & \\
     5.a & Balance on Current and Capital Accounts & \\
     5.b & Balance on Financial Accounts & \\
    \bottomrule 
    \end{tabular}
    \label{tab:my_label}
\end{table}

\begin{parts}
    \part[5] \textbf{In theory}, what should the \textbf{difference between items (5.a) and (5.b)} be?
    \vspace*{\stretch{1}}
    \part[5] What is the value of \textbf{item (5.a)} equal to? 
    Show your work.
    \vspace*{\stretch{1}}
    \part[5] What is the value of \textbf{item (5.b)} equal to? 
    Show your work.
    \vspace*{\stretch{1}}
    \part[5] What is the value of \textbf{item (4)} in this case?
    \vspace*{\stretch{1}}
    \part[5] Would this country be considered a case of CA surplus or CA deficit?
    \vspace*{\stretch{1}}
\end{parts}

\newpage 

\question[30] 
Consider a case in which a given economy reports a \textbf{GNP} level of \$5.4bn.
The \textbf{primary income net flows} are worth \$1.2bn, while \textbf{secondary income transfers} are worth \$0.2bn.

\begin{parts}
    \part[5] What is the implied level of \textbf{GDP} in this context?
    Note that primary income is associated with income flows for compensating employees and secondary income is associated with transfers of income.
    \vspace*{\stretch{1}}
    \part[5] 
    Take the following equation:
    \begin{align*}
        \text{GNP} = \text{GDP} + \text{Net Primary Income} + \text{Net Secondary Income}
    \end{align*}
    Show how the \textbf{Current Acount Suprlus} is present in the measure of \textbf{GNP}
    \vspace*{\stretch{1}}
    \part[5] Using the found value of GDP, consider the fact that \textbf{consumption} is equal to \$1.5bn, \textbf{Investment} is equal to \$1.8bn, and the government runs a balanced budget and collects \$0.4 in revenue.
    \textbf{What is the value of Net Exports?} 
    Show your work.
    \vspace*{\stretch{1}}
    \newpage 
    \part[5] What is the value of \textbf{savings}? 
    Show your work
    \vspace*{\stretch{1}}
    \part[5] Suppose there is a shock to the economy, and the government is forced to run a major budget deficit of \$1.2bn. 
    What does this imply about the \textbf{tax rate for the country}?
    \vspace*{\stretch{1}}
    \part[5] Update your measure of \textbf{GNP} to reflect this change in the government budget balance.
    What is the \textbf{percentage change in GNP}?
    Show your work. 
    \vspace*{\stretch{2}}
\end{parts}

\end{questions}

\end{document}
